\documentclass[12pt,a4paper]{article}

\usepackage[margin=2.5cm]{geometry}
\usepackage[utf8]{inputenc}
\usepackage[T1]{fontenc}
\usepackage[magyar]{babel}
\usepackage{lmodern}
\usepackage{hyperref}
\usepackage{enumitem}
\usepackage{graphicx}

\hypersetup{
    colorlinks=true,
    linkcolor=black,
    urlcolor=blue
}

\begin{document}

\begin{center}
    {\LARGE JEGYZŐKÖNYV}\\[2.5cm]

    {\large Web technológiák 1}\\[1cm]

    {\large \textbf{Egészséges Élet Gyógyszertár}}\\[6cm]
\end{center}

\hfill
\begin{minipage}{0.45\textwidth}
    Készítette: \textbf{Mizerák Bence}\\[0.2cm]
    Neptunkód: \textbf{MO0VEY}\\[0.2cm]
    Dátum: 2025. december
\end{minipage}

\vfill

\begin{center}
    Miskolc, 2025
\end{center}

\clearpage

\tableofcontents
\newpage

\section{Bevezetés}

A beadandó feladat keretében egy többoldalas, gyógyszertár témájú weboldalt készítettem.
A projekt célja az volt, hogy egy \emph{„Egészséges Élet Gyógyszertár”} nevű, fiktív
tájékoztató oldal készüljön, amely a látogató számára:
\begin{itemize}
    \item bemutatja a gyakori gyógyszereket,
    \item felsorolja a gyakori betegségeket és lehetséges kezelési módjaikat,
    \item gyógyszer nélküli életmódtanácsokat ad,
    \item egy nem valós, demonstrációs online rendelési űrlapot biztosít.
\end{itemize}

A projekt során külön hangsúlyt kapott:
\begin{itemize}
    \item az esztétikus, következetes megjelenés
    \item a használhatóság és áttekinthetőség,
\end{itemize}

\section{Projekt felépítése, főbb HTML oldalak}

A projekt könyvtárstruktúrájában a következő fő HTML oldalak találhatók:

\begin{itemize}
    \item \texttt{Gyogyszertar.html} -- főoldal,
    \item \texttt{gyogyszerek.html} -- gyógyszerlista JSON-ból,
    \item \texttt{betegsegek.html} -- betegségek ismertetése,
    \item \texttt{tanacsok.html} -- tanácsok gyógyszer nélkül + videó,
    \item \texttt{Rendeles.html} -- online rendelési űrlap (nem valós rendelés)
    
\end{itemize}

\subsection{Főoldal: \texttt{Gyogyszertar.html}}

A főoldal bevezető szerepet tölt be:
\begin{itemize}
    \item sötét, üvegszerű háttéren jelennek meg a tartalmi kártyák,
    \item a fejlécben szerepel a projekt címe: „Egészséges Élet Gyógyszertár”,
    \item rövid leírás arról, milyen információkat talál a látogató az oldalon.
\end{itemize}

A navigációs menü minden oldalon egységes:
\begin{itemize}
    \item Főoldal,
    \item Gyógyszerek,
    \item Betegségek,
    \item Tanácsok gyógyszer nélkül,
    \item Rendelés.
\end{itemize}

A főoldalon egy rövid \emph{jogi jellegű diszklémer} is található:
az itt megjelenő információk nem helyettesítik az orvosi vizsgálatot
és a gyógyszerész tanácsát.

\subsection{Gyógyszerek oldal: \texttt{gyogyszerek.html}}

Ez az oldal a gyógyszerek listáját jeleníti meg.
A tartalom nem statikusan, hanem egy külön \texttt{gyogyszertar.json} állományból töltődik be.
A JavaScript kód:

\begin{itemize}
    \item betölti a JSON-t (\texttt{fetch}),
    \item dinamikusan táblázatot hoz létre (sorok, cellák),
    \item megjeleníti a gyógyszer nevét, hatóanyagát, indikációját, kiszereléseit, árát stb.
\end{itemize}

Az oldalon lehetőség van szűrésre is (például név vagy kategória alapján),
így a látogató könnyebben megtalálhatja a számára releváns készítményeket.

\subsection{Betegségek oldal: \texttt{betegsegek.html}}

A \texttt{betegsegek.html} oldalon kártyákra bontva jelennek meg a gyakori panaszok,
például:
\begin{itemize}
    \item fejfájás,
    \item meghűlés,
    \item láz,
    \item gyomorproblémák stb.
\end{itemize}

Minden kártya tartalmaz:
\begin{itemize}
    \item rövid leírást a betegségről,
    \item lehetséges tüneteket,
    \item általános javaslatokat,
    \item súlyos tünetek esetén figyelmeztetést, hogy orvoshoz kell fordulni.
\end{itemize}

A kártyákon \emph{badge}-ek és kiemelések is vannak (pl. „enyhe tünetek”, „figyelmeztetés”),
amelyek CSS osztályokkal vannak megoldva.

\subsection{Tanácsok gyógyszer nélkül: \texttt{tanacsok.html}}

Ez az oldal kifejezetten az életmódbeli, nem gyógyszeres tanácsokra koncentrál.
Tartalmaz több, kártyaszerű blokkot, például:

\begin{itemize}
    \item alvásminőség javítása,
    \item folyadékbevitel növelése,
    \item rendszeres mozgás,
    \item stresszkezelés,
    \item digitális detox (képernyőhasználat csökkentése).
\end{itemize}

A lap külön érdekessége egy beágyazott videó:
\begin{itemize}
    \item a videó fájl: egy saját \texttt{Preventative Medicine.mp4} állomány,
    \item a videóhoz egyedi vezérlőgombokat készítettem (play, pause, némítás, hangerő és sebesség állítása),
    \item a vezérlőlogika külön \texttt{videoControls.js} fájlban található.
\end{itemize}

Ezzel a feladat azon része is teljesült, hogy legyen saját JavaScript kód, amely
egy beágyazott videót irányít, nem csak a beépített böngésző-lejátszót használjuk.

\subsection{Rendelési űrlap: \texttt{Rendeles.html}}

A \texttt{Rendeles.html} egy nem valós, demonstrációs „online rendelés” oldal.
Célja, hogy bemutassa a különféle űrlapmezők használatát és a JavaScript-alapú
validációt. Az oldal fő elemei:

\begin{itemize}
    \item Név (\texttt{text}).
    \item E-mail cím (\texttt{email}).
    \item Rendelés dátuma (\texttt{date}).
    \item Gyógyszer kiválasztása \texttt{datalist} elemmel.
    \item Kiszerelés választása rádiógombokkal.
    \item Extra opciók jelölőnégyzetekkel (pl. utánvét, e-mail visszaigazolás).
    \item Csomagolás színe (\texttt{color} input).
    \item Megjegyzés mező (\texttt{textarea}).
    \item Küldés és törlés gombok.
\end{itemize}

A gyógyszerek listája itt sem kézzel van beírva, hanem \texttt{JSON}-ból töltődik be:
a \texttt{gyogyszertar.json} fájl neveit a JavaScript húzza be a \texttt{datalist} opcióiba.

\section{Stílus és dizájn (CSS)}

A projektben több külső stíluslap is szerepel, például:

\begin{itemize}
    \item \texttt{Hatterszin.css} -- alap háttér és színek,
    \item \texttt{menu.css} -- navigációs menü stílusai,
    \item \texttt{backToTop.css} -- „vissza a tetejére” gomb megjelenése,
    \item \texttt{badges.css}, \texttt{aside.css} -- speciális kártyák és jelvények.
\end{itemize}

Az egyes oldalakon emellett belső \texttt{<style>} blokkok is találhatók,
amelyek az adott oldalra jellemző:
\begin{itemize}
    \item elrendezést (grid, flex),
    \item kártya-stílusokat (sötét üvegszerű doboz, lekerekített sarkok, árnyék),
    \item animációkat (belépő animációk, háttér finom elmozdulása),
    \item űrlap-mezők és hibaszövegek kinézetét
\end{itemize}
szabályozzák.

A teljes dizájn egységes:
sötét háttér, halvány szürke/bézs szövegszín, kékes árnyalatú kiemelések,
kártya-szerű blokkok és modern betűtípus (pl. Segoe UI / rendszer betűk).

\section{JavaScript, JSON és AJAX használata}

\subsection{Gyógyszerek JSON-ból: \texttt{gyogyszerek.js} és \texttt{gyogyszertar.json}}

A \texttt{gyogyszertar.json} állomány tartalmazza a gyógyszerek listáját.
Minden gyógyszerhez olyan adatok tartoznak, mint:
\begin{itemize}
    \item név,
    \item hatóanyag,
    \item betegségkategória,
    \item kiszerelések és árak,
    \item receptkötelesség.
\end{itemize}

A \texttt{gyogyszerek.js} feladata:
\begin{itemize}
    \item a JSON betöltése \texttt{fetch} segítségével,
    \item a JSON tömb bejárása,
    \item táblázatsorok (\texttt{<tr>}, \texttt{<td>}) dinamikus létrehozása,
    \item az adatok beírása a táblázatba,
    \item szűrési feltételek kezelése (pl. név vagy kategória alapján).
\end{itemize}

Ezzel teljesülnek a JSON+AJAX és a DOM-manipulációs követelmények is.

\subsection{Videó vezérlés: \texttt{videoControls.js}}

A \texttt{tanacsok.html} oldalba ágyazott videót egy külön
\texttt{videoControls.js} állományból irányítom. A szkript:

\begin{itemize}
    \item a \texttt{DOMContentLoaded} esemény után kiválasztja a videó elemet és a gombokat,
    \item eseménykezelőt köt a gombokra (play, pause, mute/unmute, újrakezdés),
    \item a hangerőt 0 és 1 között lépteti (kis plusz/mínusz lépésekben),
    \item a lejátszási sebességet is állíthatóvá teszi (például 0.75x, 1x, 1.25x, 1.5x).
\end{itemize}

A videó nem teljes képernyős, hanem egy elegáns \emph{„kártyában”} jelenik meg,
szöveges magyarázattal körülvéve, amely a \emph{„preventív medicina”}
(prehabilitáció, megelőzés) témáját hangsúlyozza.

\subsection{Rendelési űrlap és validáció: \texttt{rendeles.js}}

A \texttt{rendeles.js} két fő funkciót lát el:

\paragraph{1. Datalist feltöltése JSON-ból.}
\begin{itemize}
    \item A szkript betölti a \texttt{gyogyszertar.json} fájlt.
    \item Kinyeri belőle az egyedi gyógyszer-neveket.
    \item Ezekből \texttt{<option>} elemeket hoz létre, és beszúrja a
          \texttt{<datalist>} (\texttt{gyogyszer-opciok}) elemek közé.
\end{itemize}

\paragraph{2. Űrlap-validáció és hibakezelés.}
\begin{itemize}
    \item Bekéri az űrlapmezőket: név, e-mail, dátum, gyógyszer, kiszerelés.
    \item A \texttt{submit} eseményre:
          \begin{itemize}
              \item törli az előző hibajelzéseket,
              \item ellenőrzi, hogy a mezők nincsenek-e üresen,
              \item ellenőrzi az e-mail formátumát (regex),
              \item ellenőrzi, hogy van-e kiválasztott rádiógomb.
          \end{itemize}
    \item Hibák esetén:
          \begin{itemize}
              \item az érintett mező kap egy piros keretet (CSS \texttt{.hiba} osztály),
              \item alatta megjelenik egy rövid hibaüzenet (\texttt{.hiba-uzenet}).
          \end{itemize}
    \item Sikeres ellenőrzés után:
          \begin{itemize}
              \item egy felugró \texttt{alert} köszöni meg a „rendelést”,
              \item az űrlap tartalma törlődik (\texttt{form.reset()}).
          \end{itemize}
\end{itemize}

Fontos, hogy az űrlap nem küld valódi adatot a szerver felé, teljesen kliensoldali demonstráció.

\section{jQuery használata és animáció}

A jQuery-t a főoldalon, a \texttt{Gyogyszertar.html}-ben használom.
A jQuery CDN a \texttt{<head>} részben töltődik be, majd az oldal alján egy rövid
szkript gondoskodik az egyszerű, látványos animációról:

\begin{verbatim}
$(function () {
    $("main").hide().fadeIn(2000);
});
\end{verbatim}

Ebben:

\begin{itemize}
    \item \texttt{\$("main")} -- jQuery-s \emph{elemkiválasztás}, a \texttt{<main>} tartalmi részt választja ki,
    \item \texttt{hide().fadeIn(2000)} -- jQuery-s \emph{animáció}:
          a \texttt{main} elemet elrejti, majd 2000 ms (2 másodperc) alatt finoman megjeleníti.
\end{itemize}

Ez jól demonstrálja a jQuery legalapvetőbb funkcióit:
\begin{itemize}
    \item elem kiválasztása CSS-szelektorral,
    \item láncolt metódusok használata,
    \item egyszerű animációk készítése.
\end{itemize}



\end{document}